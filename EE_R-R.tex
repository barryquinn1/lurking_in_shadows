% Options for packages loaded elsewhere
\PassOptionsToPackage{unicode}{hyperref}
\PassOptionsToPackage{hyphens}{url}
\PassOptionsToPackage{dvipsnames,svgnames,x11names}{xcolor}
%
\documentclass[
  letterpaper,
  DIV=11,
  numbers=noendperiod]{scrartcl}
\usepackage{amsmath,amssymb}
\usepackage{lmodern}
\usepackage{iftex}
\ifPDFTeX
  \usepackage[T1]{fontenc}
  \usepackage[utf8]{inputenc}
  \usepackage{textcomp} % provide euro and other symbols
\else % if luatex or xetex
  \usepackage{unicode-math}
  \defaultfontfeatures{Scale=MatchLowercase}
  \defaultfontfeatures[\rmfamily]{Ligatures=TeX,Scale=1}
\fi
% Use upquote if available, for straight quotes in verbatim environments
\IfFileExists{upquote.sty}{\usepackage{upquote}}{}
\IfFileExists{microtype.sty}{% use microtype if available
  \usepackage[]{microtype}
  \UseMicrotypeSet[protrusion]{basicmath} % disable protrusion for tt fonts
}{}
\makeatletter
\@ifundefined{KOMAClassName}{% if non-KOMA class
  \IfFileExists{parskip.sty}{%
    \usepackage{parskip}
  }{% else
    \setlength{\parindent}{0pt}
    \setlength{\parskip}{6pt plus 2pt minus 1pt}}
}{% if KOMA class
  \KOMAoptions{parskip=half}}
\makeatother
\usepackage{xcolor}
\usepackage{color}
\usepackage{fancyvrb}
\newcommand{\VerbBar}{|}
\newcommand{\VERB}{\Verb[commandchars=\\\{\}]}
\DefineVerbatimEnvironment{Highlighting}{Verbatim}{commandchars=\\\{\}}
% Add ',fontsize=\small' for more characters per line
\usepackage{framed}
\definecolor{shadecolor}{RGB}{241,243,245}
\newenvironment{Shaded}{\begin{snugshade}}{\end{snugshade}}
\newcommand{\AlertTok}[1]{\textcolor[rgb]{0.68,0.00,0.00}{#1}}
\newcommand{\AnnotationTok}[1]{\textcolor[rgb]{0.37,0.37,0.37}{#1}}
\newcommand{\AttributeTok}[1]{\textcolor[rgb]{0.40,0.46,0.14}{#1}}
\newcommand{\BaseNTok}[1]{\textcolor[rgb]{0.68,0.00,0.00}{#1}}
\newcommand{\BuiltInTok}[1]{\textcolor[rgb]{0.00,0.46,0.62}{#1}}
\newcommand{\CharTok}[1]{\textcolor[rgb]{0.13,0.47,0.30}{#1}}
\newcommand{\CommentTok}[1]{\textcolor[rgb]{0.37,0.37,0.37}{#1}}
\newcommand{\CommentVarTok}[1]{\textcolor[rgb]{0.37,0.37,0.37}{\textit{#1}}}
\newcommand{\ConstantTok}[1]{\textcolor[rgb]{0.56,0.35,0.01}{#1}}
\newcommand{\ControlFlowTok}[1]{\textcolor[rgb]{0.00,0.46,0.62}{#1}}
\newcommand{\DataTypeTok}[1]{\textcolor[rgb]{0.68,0.00,0.00}{#1}}
\newcommand{\DecValTok}[1]{\textcolor[rgb]{0.68,0.00,0.00}{#1}}
\newcommand{\DocumentationTok}[1]{\textcolor[rgb]{0.37,0.37,0.37}{\textit{#1}}}
\newcommand{\ErrorTok}[1]{\textcolor[rgb]{0.68,0.00,0.00}{#1}}
\newcommand{\ExtensionTok}[1]{\textcolor[rgb]{0.00,0.46,0.62}{#1}}
\newcommand{\FloatTok}[1]{\textcolor[rgb]{0.68,0.00,0.00}{#1}}
\newcommand{\FunctionTok}[1]{\textcolor[rgb]{0.28,0.35,0.67}{#1}}
\newcommand{\ImportTok}[1]{\textcolor[rgb]{0.00,0.46,0.62}{#1}}
\newcommand{\InformationTok}[1]{\textcolor[rgb]{0.37,0.37,0.37}{#1}}
\newcommand{\KeywordTok}[1]{\textcolor[rgb]{0.00,0.46,0.62}{#1}}
\newcommand{\NormalTok}[1]{\textcolor[rgb]{0.00,0.46,0.62}{#1}}
\newcommand{\OperatorTok}[1]{\textcolor[rgb]{0.37,0.37,0.37}{#1}}
\newcommand{\OtherTok}[1]{\textcolor[rgb]{0.00,0.46,0.62}{#1}}
\newcommand{\PreprocessorTok}[1]{\textcolor[rgb]{0.68,0.00,0.00}{#1}}
\newcommand{\RegionMarkerTok}[1]{\textcolor[rgb]{0.00,0.46,0.62}{#1}}
\newcommand{\SpecialCharTok}[1]{\textcolor[rgb]{0.37,0.37,0.37}{#1}}
\newcommand{\SpecialStringTok}[1]{\textcolor[rgb]{0.13,0.47,0.30}{#1}}
\newcommand{\StringTok}[1]{\textcolor[rgb]{0.13,0.47,0.30}{#1}}
\newcommand{\VariableTok}[1]{\textcolor[rgb]{0.07,0.07,0.07}{#1}}
\newcommand{\VerbatimStringTok}[1]{\textcolor[rgb]{0.13,0.47,0.30}{#1}}
\newcommand{\WarningTok}[1]{\textcolor[rgb]{0.37,0.37,0.37}{\textit{#1}}}
\usepackage{longtable,booktabs,array}
\usepackage{calc} % for calculating minipage widths
% Correct order of tables after \paragraph or \subparagraph
\usepackage{etoolbox}
\makeatletter
\patchcmd\longtable{\par}{\if@noskipsec\mbox{}\fi\par}{}{}
\makeatother
% Allow footnotes in longtable head/foot
\IfFileExists{footnotehyper.sty}{\usepackage{footnotehyper}}{\usepackage{footnote}}
\makesavenoteenv{longtable}
\usepackage{graphicx}
\makeatletter
\def\maxwidth{\ifdim\Gin@nat@width>\linewidth\linewidth\else\Gin@nat@width\fi}
\def\maxheight{\ifdim\Gin@nat@height>\textheight\textheight\else\Gin@nat@height\fi}
\makeatother
% Scale images if necessary, so that they will not overflow the page
% margins by default, and it is still possible to overwrite the defaults
% using explicit options in \includegraphics[width, height, ...]{}
\setkeys{Gin}{width=\maxwidth,height=\maxheight,keepaspectratio}
% Set default figure placement to htbp
\makeatletter
\def\fps@figure{htbp}
\makeatother
\setlength{\emergencystretch}{3em} % prevent overfull lines
\providecommand{\tightlist}{%
  \setlength{\itemsep}{0pt}\setlength{\parskip}{0pt}}
\setcounter{secnumdepth}{-\maxdimen} % remove section numbering
% Make \paragraph and \subparagraph free-standing
\ifx\paragraph\undefined\else
  \let\oldparagraph\paragraph
  \renewcommand{\paragraph}[1]{\oldparagraph{#1}\mbox{}}
\fi
\ifx\subparagraph\undefined\else
  \let\oldsubparagraph\subparagraph
  \renewcommand{\subparagraph}[1]{\oldsubparagraph{#1}\mbox{}}
\fi
\KOMAoption{captions}{tableheading}
\makeatletter
\makeatother
\makeatletter
\@ifpackageloaded{caption}{}{\usepackage{caption}}
\AtBeginDocument{%
\renewcommand*\contentsname{Table of contents}
\renewcommand*\listfigurename{List of Figures}
\renewcommand*\listtablename{List of Tables}
\renewcommand*\figurename{Figure}
\renewcommand*\tablename{Table}
}
\@ifpackageloaded{float}{}{\usepackage{float}}
\floatstyle{ruled}
\@ifundefined{c@chapter}{\newfloat{codelisting}{h}{lop}}{\newfloat{codelisting}{h}{lop}[chapter]}
\floatname{codelisting}{Listing}
\newcommand*\listoflistings{\listof{codelisting}{List of Listings}}
\makeatother
\makeatletter
\@ifpackageloaded{caption}{}{\usepackage{caption}}
\@ifpackageloaded{subcaption}{}{\usepackage{subcaption}}
\makeatother
\makeatletter
\@ifpackageloaded{tcolorbox}{}{\usepackage[many]{tcolorbox}}
\makeatother
\makeatletter
\@ifundefined{shadecolor}{\definecolor{shadecolor}{rgb}{.97, .97, .97}}
\makeatother
\makeatletter
\makeatother
\ifLuaTeX
  \usepackage{selnolig}  % disable illegal ligatures
\fi
\IfFileExists{bookmark.sty}{\usepackage{bookmark}}{\usepackage{hyperref}}
\IfFileExists{xurl.sty}{\usepackage{xurl}}{} % add URL line breaks if available
\urlstyle{same} % disable monospaced font for URLs
\hypersetup{
  pdftitle={Lurking in the shadows: The impact of CO2 emissions target setting on carbon pricing and environmental efficiency.},
  pdfauthor={Barry Quinn (Queen's Management School); Ronan Gallagher (University of Edinburgh Business School); Timo Kuosmanen (Aalto Business School)},
  colorlinks=true,
  linkcolor={blue},
  filecolor={Maroon},
  citecolor={Blue},
  urlcolor={Blue},
  pdfcreator={LaTeX via pandoc}}

\title{\emph{Lurking in the shadows:} The impact of CO\textsubscript{2}
emissions target setting on carbon pricing and environmental
efficiency.}
\author{Barry Quinn (Queen's Management School) \and Ronan Gallagher
(University of Edinburgh Business School) \and Timo Kuosmanen (Aalto
Business School)}
\date{}

\begin{document}
\maketitle

\ifdefined\Shaded\renewenvironment{Shaded}{\begin{tcolorbox}[boxrule=0pt, sharp corners, interior hidden, borderline west={3pt}{0pt}{shadecolor}, enhanced, frame hidden, breakable]}{\end{tcolorbox}}\fi

\hypertarget{abstract}{%
\subsection{Abstract}\label{abstract}}

This paper studies the impact of CO\textsubscript{2} emissions target
setting. We empirically investigate the targets set during the Kyoto
Protocol period using a convex non-parametric least squares system,
quantile regressions, and a comprehensive data set of 125 countries. Our
findings reveal CO\textsubscript{2} marginal abatement costs, which: (1)
are significantly higher for target setting countries; (2) increase over
the sample period; (3) and are an order of magnitude greater than the
prevailing emissions pricing mechanisms. The results provide insights
into the consequences of policies to curb unwanted by-products in a
regulated system and shed light on the price efficiency of carbon
markets. Furthermore, we contribute to the debate on emission reduction
standard-setting and highlight the importance of shadow price estimates
when regulating market instabilities in an emission trading scheme.

\hypertarget{introduction}{%
\subsection{Introduction}\label{introduction}}

The World Health Organization predicts significant health risks
associated with climate change. Their analysis estimates around 250,000
additional deaths per year from 2030 to 2050, assuming the status quo of
current abatement practices and global economic growth\footnote{These
  statistics are taken from the WHO factsheet on climate change and
  health
  \url{https://www.who.int/news-room/fact-sheets/detail/climate-change-and-health}}.
The reduction of greenhouse emissions and the impact on climate change
are existential challenges of the 21st century. Many countries have
adopted emissions reductions targets since the dawn of the Kyoto
Protocol (hereafter KP)\footnote{The KP was principled on the idea of
  hard targets for emissions reduction for industrialised nations and
  the EU. These were developed in tandem with carbon trading mechanisms,
  the largest of which is the EU Emissions Trading System (EU-ETS).
  These developments brought not just matters of environmental
  production efficiency to the fore but also those relating to carbon
  price discovery.} in response to this challenge. The nature and
efficacy in response to this challenge of these targets have attracted
considerable conceptual debate (for example, see Angelis, Di Giacomo,
and Vannoni 2019). Still, there are scant empirical studies on the
impact of explicit KP target setting. We attempt to solve this puzzle by
testing the differences between target-setting and non-target-setting
countries. We use an identification strategy that more accurately
estimates the impact of target setting on carbon pricing and
environmental efficiency. Specifically, we focus on the KP target
setting period and analyse CO2 emissions for 125. countries. Considering
only CO2. emissions allow for a representative sample of non-target
setting Countries (non-annexed 1 Countries) and more meaningful group
comparisons in our statistical tests. We use convex quantile regression
(CQR) to estimate shadow prices (see equation (1))(Kuosmanen and Zhou
2021a) and an improved marginal abatement cost (MAC) of CO2 emissions
(Xian et al.~2022; Dai, Zhou, and Kuosmanen 2020; Kuosmanen, Zhou, and
Dai 2020a, 2020b). Moreover, our method allows for an examination of the
factors which help explain relative (in)efficiencies.

We find that target setters during the first KP commitment period were
more environmentally inefficient than non-target setters, an unintended
consequence of the regulation. We also note that countries with a higher
degree of industrialisation and those with more urban populations
exhibit lower environmental efficiency. Our results also assert that the
marginal cost of CO2 reduction during the first KP period was an order
of magnitude higher than the trading price of CO2 in the EU-ETS. This
result suggests considerable price inefficiency in the emissions market.

Our findings have important implications for international carbon
regulation. Authors have highlighted potential gains in CO2 mitigation
from emission trading schemes (for example, Kumar, Managi, and Jain
2020). Our findings add to this debate. We show that the shadow prices
and market prices of CO2 diverge in the KP period, suggestive of a
consistent misallocation of the traded allowances in the EU emissions
trading scheme (ETS). Our results also show an imbalance in shadow
pricing due to target setting. These significant frictions in the price
discovery of an ETS market may result in a surplus of allowances,
exacerbating market instability, and a lower carbon price. The latter
likely weakened the incentives to lower emissions. We argue that when
policymakers debate structural measures to promote market stability,
such as predefined rules to place unallocated allowances in a market
stability reserve, shadow price imbalances due to target setting must be
considered{[}3{]}.

In the next section, we review the literature on the impact of the KP on
emissions and productive efficiency. Next, we describe the frontier
models and data used. We follow with a discussion of our findings and
conclusions.

\hypertarget{running-code}{%
\subsection{Running Code}\label{running-code}}

When you click the \textbf{Render} button a document will be generated
that includes both content and the output of embedded code. You can
embed code like this:

\begin{Shaded}
\begin{Highlighting}[]
\DecValTok{1} \SpecialCharTok{+} \DecValTok{1}
\end{Highlighting}
\end{Shaded}

\begin{verbatim}
[1] 2
\end{verbatim}

You can add options to executable code like this

\begin{verbatim}
[1] 4
\end{verbatim}

The \texttt{echo:\ false} option disables the printing of code (only
output is displayed).

\end{document}
