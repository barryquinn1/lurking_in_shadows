% Options for packages loaded elsewhere
\PassOptionsToPackage{unicode}{hyperref}
\PassOptionsToPackage{hyphens}{url}
\PassOptionsToPackage{dvipsnames,svgnames,x11names}{xcolor}
%
\documentclass[
  letterpaper,
  DIV=11,
  numbers=noendperiod]{scrartcl}
\usepackage{amsmath,amssymb}
\usepackage{lmodern}
\usepackage{iftex}
\ifPDFTeX
  \usepackage[T1]{fontenc}
  \usepackage[utf8]{inputenc}
  \usepackage{textcomp} % provide euro and other symbols
\else % if luatex or xetex
  \usepackage{unicode-math}
  \defaultfontfeatures{Scale=MatchLowercase}
  \defaultfontfeatures[\rmfamily]{Ligatures=TeX,Scale=1}
\fi
% Use upquote if available, for straight quotes in verbatim environments
\IfFileExists{upquote.sty}{\usepackage{upquote}}{}
\IfFileExists{microtype.sty}{% use microtype if available
  \usepackage[]{microtype}
  \UseMicrotypeSet[protrusion]{basicmath} % disable protrusion for tt fonts
}{}
\makeatletter
\@ifundefined{KOMAClassName}{% if non-KOMA class
  \IfFileExists{parskip.sty}{%
    \usepackage{parskip}
  }{% else
    \setlength{\parindent}{0pt}
    \setlength{\parskip}{6pt plus 2pt minus 1pt}}
}{% if KOMA class
  \KOMAoptions{parskip=half}}
\makeatother
\usepackage{xcolor}
\IfFileExists{xurl.sty}{\usepackage{xurl}}{} % add URL line breaks if available
\IfFileExists{bookmark.sty}{\usepackage{bookmark}}{\usepackage{hyperref}}
\hypersetup{
  pdftitle={Lurking in the shadows: The impact of CO2 emissions target setting on carbon pricing and environmental efficiency.},
  pdfauthor={Barry Quinn (Queen's Management School); Ronan Gallagher (University of Edinburgh Business School); Timo Kuosmanen (Aalto Business School)},
  colorlinks=true,
  linkcolor={blue},
  filecolor={Maroon},
  citecolor={Blue},
  urlcolor={Blue},
  pdfcreator={LaTeX via pandoc}}
\urlstyle{same} % disable monospaced font for URLs
\setlength{\emergencystretch}{3em} % prevent overfull lines
\setcounter{secnumdepth}{-\maxdimen} % remove section numbering
% Make \paragraph and \subparagraph free-standing
\ifx\paragraph\undefined\else
  \let\oldparagraph\paragraph
  \renewcommand{\paragraph}[1]{\oldparagraph{#1}\mbox{}}
\fi
\ifx\subparagraph\undefined\else
  \let\oldsubparagraph\subparagraph
  \renewcommand{\subparagraph}[1]{\oldsubparagraph{#1}\mbox{}}
\fi


\providecommand{\tightlist}{%
  \setlength{\itemsep}{0pt}\setlength{\parskip}{0pt}}\usepackage{longtable,booktabs,array}
\usepackage{calc} % for calculating minipage widths
% Correct order of tables after \paragraph or \subparagraph
\usepackage{etoolbox}
\makeatletter
\patchcmd\longtable{\par}{\if@noskipsec\mbox{}\fi\par}{}{}
\makeatother
% Allow footnotes in longtable head/foot
\IfFileExists{footnotehyper.sty}{\usepackage{footnotehyper}}{\usepackage{footnote}}
\makesavenoteenv{longtable}
\usepackage{graphicx}
\makeatletter
\def\maxwidth{\ifdim\Gin@nat@width>\linewidth\linewidth\else\Gin@nat@width\fi}
\def\maxheight{\ifdim\Gin@nat@height>\textheight\textheight\else\Gin@nat@height\fi}
\makeatother
% Scale images if necessary, so that they will not overflow the page
% margins by default, and it is still possible to overwrite the defaults
% using explicit options in \includegraphics[width, height, ...]{}
\setkeys{Gin}{width=\maxwidth,height=\maxheight,keepaspectratio}
% Set default figure placement to htbp
\makeatletter
\def\fps@figure{htbp}
\makeatother
\newlength{\cslhangindent}
\setlength{\cslhangindent}{1.5em}
\newlength{\csllabelwidth}
\setlength{\csllabelwidth}{3em}
\newlength{\cslentryspacingunit} % times entry-spacing
\setlength{\cslentryspacingunit}{\parskip}
\newenvironment{CSLReferences}[2] % #1 hanging-ident, #2 entry spacing
 {% don't indent paragraphs
  \setlength{\parindent}{0pt}
  % turn on hanging indent if param 1 is 1
  \ifodd #1
  \let\oldpar\par
  \def\par{\hangindent=\cslhangindent\oldpar}
  \fi
  % set entry spacing
  \setlength{\parskip}{#2\cslentryspacingunit}
 }%
 {}
\usepackage{calc}
\newcommand{\CSLBlock}[1]{#1\hfill\break}
\newcommand{\CSLLeftMargin}[1]{\parbox[t]{\csllabelwidth}{#1}}
\newcommand{\CSLRightInline}[1]{\parbox[t]{\linewidth - \csllabelwidth}{#1}\break}
\newcommand{\CSLIndent}[1]{\hspace{\cslhangindent}#1}

\KOMAoption{captions}{tableheading}
\makeatletter
\makeatother
\makeatletter
\@ifpackageloaded{caption}{}{\usepackage{caption}}
\AtBeginDocument{%
\ifdefined\contentsname
  \renewcommand*\contentsname{Table of contents}
\else
  \newcommand\contentsname{Table of contents}
\fi
\ifdefined\listfigurename
  \renewcommand*\listfigurename{List of Figures}
\else
  \newcommand\listfigurename{List of Figures}
\fi
\ifdefined\listtablename
  \renewcommand*\listtablename{List of Tables}
\else
  \newcommand\listtablename{List of Tables}
\fi
\ifdefined\figurename
  \renewcommand*\figurename{Figure}
\else
  \newcommand\figurename{Figure}
\fi
\ifdefined\tablename
  \renewcommand*\tablename{Table}
\else
  \newcommand\tablename{Table}
\fi
}
\@ifpackageloaded{float}{}{\usepackage{float}}
\floatstyle{ruled}
\@ifundefined{c@chapter}{\newfloat{codelisting}{h}{lop}}{\newfloat{codelisting}{h}{lop}[chapter]}
\floatname{codelisting}{Listing}
\newcommand*\listoflistings{\listof{codelisting}{List of Listings}}
\makeatother
\makeatletter
\@ifpackageloaded{caption}{}{\usepackage{caption}}
\@ifpackageloaded{subcaption}{}{\usepackage{subcaption}}
\makeatother
\makeatletter
\@ifpackageloaded{tcolorbox}{}{\usepackage[many]{tcolorbox}}
\makeatother
\makeatletter
\@ifundefined{shadecolor}{\definecolor{shadecolor}{rgb}{.97, .97, .97}}
\makeatother
\makeatletter
\makeatother
\ifLuaTeX
  \usepackage{selnolig}  % disable illegal ligatures
\fi

\title{\emph{Lurking in the shadows:} The impact of CO\textsubscript{2}
emissions target setting on carbon pricing and environmental
efficiency.}
\author{Barry Quinn (Queen's Management School) \and Ronan Gallagher
(University of Edinburgh Business School) \and Timo Kuosmanen (Aalto
Business School)}
\date{}

\begin{document}
\maketitle

\ifdefined\Shaded\renewenvironment{Shaded}{\begin{tcolorbox}[interior hidden, sharp corners, boxrule=0pt, breakable, frame hidden, borderline west={3pt}{0pt}{shadecolor}, enhanced]}{\end{tcolorbox}}\fi

\hypertarget{abstract}{%
\section{Abstract}\label{abstract}}

This paper studies the impact of CO\textsubscript{2} emissions target
setting. We empirically investigate the targets set during the Kyoto
Protocol period using a convex non-parametric least squares system,
quantile regressions, and a comprehensive data set of 125 countries. Our
findings reveal CO\textsubscript{2} marginal abatement costs, which: (1)
are significantly higher for target setting countries; (2) increase over
the sample period; (3) and are an order of magnitude greater than the
prevailing emissions pricing mechanisms. The results provide insights
into the consequences of policies to curb unwanted by-products in a
regulated system and shed light on the price efficiency of carbon
markets. Furthermore, we contribute to the debate on emission reduction
standard-setting and highlight the importance of shadow price estimates
when regulating market instabilities in an emission trading scheme.

\newpage

\hypertarget{introduction}{%
\section{Introduction}\label{introduction}}

The World Health Organization predicts significant health risks
associated with climate change. Their analysis estimates around 250,000
additional deaths per year from 2030 to 2050, assuming the status quo of
current abatement practices and global economic growth\footnote{These
  statistics are taken from the WHO factsheet on climate change and
  health
  \url{https://www.who.int/news-room/fact-sheets/detail/climate-change-and-health}}.
The reduction of greenhouse emissions and the impact on climate change
are existential challenges of the 21st century. Many countries have
adopted emissions reductions targets since the dawn of the Kyoto
Protocol (hereafter KP)\footnote{The KP was principled on the idea of
  hard targets for emissions reduction for industrialised nations and
  the EU. These were developed in tandem with carbon trading mechanisms,
  the largest of which is the EU Emissions Trading System (EU-ETS).
  These developments brought not just matters of environmental
  production efficiency to the fore but also those relating to carbon
  price discovery.} in response to this challenge. The nature and
efficacy in response to this challenge of these targets have attracted
considerable conceptual debate (for example, see Angelis, Di Giacomo,
and Vannoni (2019) ), but few empirical studies on the impact of
explicit target setting.

We attempt to solve this puzzle by testing the differences between
target-setting and non-target-setting countries. We use an
identification strategy that more accurately estimates the impact of
target setting on carbon pricing and environmental efficiency.
Specifically, we focus on the KP target setting period and analyse
CO\textsubscript{2} emissions for 125 countries. Considering only
CO\textsubscript{2} emissions allow for a representative sample of
non-target setting Countries (non-annexed 1 Countries) and more
meaningful group comparisons in our statistical tests. We use quantile
system of convex nonparametric least squares regressions (CQR) to
estimate shadow prices (see equation (1) in Kuosmanen and Zhou (2021a) )
and an improved marginal abatement cost (MAC) of CO\textsubscript{2}
emissions (Xian et al. 2022; Dai, Zhou, and Kuosmanen 2020; Kuosmanen,
Zhou, and Dai 2020; Kuosmanen and Zhou 2021b). Convex nonparametric
least squares has recently been found to admit a causal interpretation
between inefficiency and productivity (Tsionas 2022). Moreover, our
method allows for an examination of the factors which help explain
relative (in)efficiencies.

We find that target setters during the first KP commitment period were
more environmentally inefficient than non-target setters, an unintended
consequence of the regulation. We also note that countries with a higher
degree of industrialisation and those with more urban populations
exhibit lower environmental efficiency. Our results also assert that the
marginal cost of CO\textsubscript{2} reduction during the first KP
period was an order of magnitude higher than the trading price of
CO\textsubscript{2} in the EU-ETS. This result suggests considerable
price inefficiency in the emissions market.

Our findings have important implications for international carbon
regulation. Authors have highlighted potential gains in
CO\textsubscript{2} mitigation from emission trading schemes (for
example, Kumar, Managi, and Jain (2020) ). Our findings add to this
debate. We show that the shadow prices and market prices of
CO\textsubscript{2} diverge in the KP period, suggestive of a consistent
misallocation of the traded allowances in the EU emissions trading
scheme (ETS). Our results also show an imbalance in shadow pricing due
to target setting. These significant frictions in the price discovery of
an ETS market may result in a surplus of allowances, exacerbating market
instability, and a lower carbon price. The latter likely weakened the
incentives to lower emissions. We argue that when policymakers debate
structural measures to promote market stability, such as predefined
rules to place unallocated allowances in a market stability reserve,
shadow price imbalances due to target setting must be considered{[}3{]}.

In the next section, we review the literature on the impact of the KP on
emissions and productive efficiency. Next, we describe the frontier
models and data used. We follow with a discussion of our findings and
conclusions.

\hypertarget{literature-review}{%
\section{Literature Review}\label{literature-review}}

The academic inquiry into the effective management of climate change has
a rich history. Historically, holistic models seek to understand how
human development, societal choices, and the natural world integrate and
influence each other. At a simplistic level, they can estimate the
social cost of carbon pollutants. This top-down approach to the
economics of climate change has been at the forefront of the discipline
(Vale 2016). However, such a global approach may prove dated in the face
of stalled international coordination on climate change policy.

Against the bedrock of climate science, the KP agreement was an
ambitious attempt to coordinate across borders on targets for emissions
reduction. The KP set out to differentiate reduction targets equitably
in terms of a nation's industrial development, a comparable level of
pollution, and the ability to mitigate the ecological damage of global
emissions levels. Specifically, countries were categorised into two
Annexes. Annexe 2 countries, which set explicit targets, were mostly
developed nations, with higher industrial production. Annexe 1
countries, defined as developing, were not subject to targets, although
most ratified the Protocol.

In the run-up to the end of the first commitment period of the KP, there
were political moves to create second commitment period targets. The
Doha amendment in 2012 extended the scope of the protocol targets to
cover the period until 2020. The Doha Amendment was a bridge arrangement
up to 2020 until a new global agreement; the Paris Agreement came into
force. The Paris agreement has attracted considerable criticism.
Commentators cite a lack of explicit targets, ambiguity in regard to
sanctions for failing to meet targets, and a more explicit international
focus as a critical weakness to country policymakers taking direct
ownership of their emissions targets.

As global cooperation has stalled in the last decade, attention in the
policy debate has shifted towards bottom-up strategies for climate
change mitigation. Vale (2016) argues that the lack of collective
political will, in turn, has shifted the nature of the associated
academic enquiry. The recent focus on the economics of catastrophic risk
insurance, trade and climate, and climate change adaptation represents a
shift towards a more realistic investigation of climate policy in an age
where the globally coordinated climate action seems illusory.

A common approach to establishing world-wide cost estimates for the
reduction of polluting emissions is to establish a global production
model. In such a model, based on a set of capital and human inputs, each
country acts as a producer of desirable outputs such as GDP at the cost
of producing undesirable outputs such as pollutants from industrial
activity. Under various assumptions, such a model can reveal each
country's relative (in)efficiency,calibrated against a backdrop of
global optimal environmental efficiency. Furthermore, abatement of
pollution output can be at the marginal cost of the desirable output
foregone.

Zhang and Folmer (1998) document and critique the myriad of marginal
abatement cost models. They consider both bottom-up technology-based
models and top-down macroeconomic models. They conclude that combining
these models best assesses the overall consequences of controlling CO2
emissions. Nordhaus and Boyer (1999) use a scenario-based approach to
analyse the economics of various trading emission schemes (ETS) for
Annex I countries for the KP. They find costs of the ETS's are seven
times greater than the benefits, two/thirds of the net global cost of
\$716 billion, are borne by the US{[}\^{}4{]} and conclude that the
proposed schemes are highly cost-ineffective.

{[}\^{}4{]}: Compared to a *so-called'' efficient abatement strategy for
global temperature reduction, the proposed strategy was eight times more
costly.

This early work was suggestive of a broad approach to abatement cost
analysis beyond the consideration of CO\textsuperscript{2} associated
pollution. Reilly et al. (1999) use the Regional Integrated Model of
Climate and Economy (RICE) to show that a multi-gas control strategy
could significantly reduce the costs of fulfilling the KP compared with
a CO\textsuperscript{2}-only strategy. They argue that the global
warming mitigation potential of the KP is limited and argue for a more
comprehensive multi-gas approach. Burniaux (2000) extend previous OECD
analysis to emission abatement of methane and nitrous oxide. They
conclude that the economic costs of implementing the targets in the KP
are lower than suggested by previous CO\textsuperscript{2}-only results.
In the longer term, most abatement will likely have to come from
CO\textsuperscript{2}, and the inclusion of other gases in the analysis
may not substantially alter estimates of economic costs.

In the later years of the KP period, researchers considered a more
statistically sophisticated approach for analysing the KP. Buonanno,
Carraro, and Galeotti (2003) adapt the RICE integrated assessment model
to account for endogenous technical change{[}\^{}5{]} and shows that
results are significantly impacted when modelling R\&D. They find that
total costs of compliance with Kyoto; are higher with induced technical
change; are reduced when trading permits are introduced, and
technological spillover reduces the incentive for R\&D, but overall
costs are higher in the presence of spillovers. McKibbin and Wilcoxen
(2004) update their earlier estimates of the cost of the KP using the
G-Cubed model, taking into account the new sink allowances from recent
negotiations as well as allowing for multiple gases and new land
clearing estimates. They perform a sensitivity analysis of compliance
costs to unexpected changes in future economic conditions. The paper
evaluates the policies under two plausible alternative assumptions about
a single aspect of the future world economy: the rate of productivity
growth in Russia. They find moderate growth in Russia would raise the
cost of the KP by as much as fifty per cent but would have little effect
on the cost of the alternative policy. They conclude that the KP is
inherently unstable because unexpected future events could raise
compliance costs substantially and place enormous pressure on
governments to rescind the agreement. The alternative policy would be
far more stable because it does not subject future governments to
adverse shocks in compliance costs. Fischer and Morgenstern (2006) find
that estimates of marginal abatement costs for reducing carbon emissions
in the United States by the significant economic-energy models vary by a
factor of five, undermining support for mandatory policies to reduce
greenhouse gas emissions. Their meta-analysis explains which modelling
assumptions are most important for understanding these cost differences
and argues for developing more consistent modelling practices for policy
analysis.

{[}\^{}5{]}: They explore three formulations; technical change is
endogenous and enters the production function via the domestic stock of
knowledge; there is an additional effect of domestic stock of knowledge
on the emission-output ratio; the output of domestic R\&D spills over
the other regions' productivity and emission-output ratio.

In more recent studies researchers focus on how a country's economic
characteristics fluctuate with abatement challenges. Halkos and Tzeremes
(2014) apply a probabilistic DEA approach to estimate conditional and
unconditional environmental efficiency of 110 countries in 2007. They
find that a country's environmental efficiency is influenced in a
non-linear mannerby both the obliged percentage levels of emission
reductions and the duration forwhich a country has signed the KP. Cifci
and Oliver (2018) use regression techniques to illustrate the
conflicting political strands of the climate change argument. The
results show that the KP reduced Annex I countries' GHG emissions by
approximately 1 million metric tons of CO2 equivalent relative to
non-Annex I countries. Contrariwise, these countries experienced an
average reduction in GDP per capita growth of 1-2 per cent relative to
non-Annex I countries. Both findings illustrate that the international
climate change agreements are fragile due to the clash of short-term
political goals with long-term reduction ambitions.

\hypertarget{empirical-design}{%
\section{Empirical Design}\label{empirical-design}}

We model the global economy as a production machine. Capital and labour
are inputs, creating economic output (desirable). However in doing so
the economic ``machine'' also produces CO2 emissions which are
undesirable. We use the concept of an efficiency frontier to model
combinations of inputs and outputs. We do so not in a deterministic way,
but using stochastic non-parametric methods. This affords us the
advantage of not having to specify a functional form of the input-output
relationship a priori and also the ability to model noise in the data.
We are concerned both with measuring environmental inefficiency
(distance from the frontier), but perhaps more subtly shadow costs.
These shadow costs can be interpreted as opportunity costs which allow
us to price CO2 emissions, or put differently to calculate the marginal
cost of CO2 abatement.

The primary focus of our analysis is shadow prices estimation for
CO\textsuperscript{2} emissions from fossil fuels. Previous studies have
provided inaccurate measures as a result of several methodological
shortcomings including:

* only considering downscaling of production and not increases in input
use.

* measuring estimates on the frontier, ignoring the actual level of
performance.

* deterministic estimation, which explicitly ignores the impact of noise
in the data.

These factors lead to overestimation of both shadow prices and group
differences in shadow prices between target setting and non-target
setting countries. Our study uses convex quantile regression methods to
estimate local approximations of shadow prices calibrated using observed
inefficiencies. Specifically, we exploit the CQR framework in Kuosmanen
and Zhou (2021a) to estimate shadow prices at observed performance
levels. Importantly, this approach is robust to the observed
heterogeneity, the choice of direction vector and accommodates
noise-based uncertainty. The following linear programming problem is
solved to estimate the distance function:

\begin{equation}\protect\hypertarget{eq-1}{}{
\begin{equation}\begin{split}& \underset{\alpha,\beta,\gamma,\delta,\epsilon^-,\epsilon^+}{\text{min}} (1-\tau) \sum^{T}_{t=1} \sum^{n}_{i=1}\epsilon^-_{it} + \tau \sum^{T}_{t=1}  \sum^{n}_{i=1}\epsilon^+_{it}  \\&\text{s.t.} \\&\gamma^{'}_{it}y_{it}=\alpha_{it}+\beta^{'}_{it}x_{it}+\delta^{'}_{it}b_{it} + \omega Z_{it} -\epsilon^-_{it}+\epsilon^+_{it} \; \forall i ,\forall t \\&\alpha_{it}+\beta^{'}_{it}x_{it}+\delta^{'}_{it}b_{it}-\gamma^{'}_{it}y_{it} \leq \alpha_{hs}+\beta^{'}_{hs}x_{it}+\delta^{'}_{hs}b_{it}-\gamma^{'}_{hs}y_{it} \; \forall i,h ; \forall t,s \\& \beta^{'}_{it}g^x+\delta^{'}_{it}g^b+\gamma^{'}_{it}g^y=1 \; \forall i,t\\& \beta_{it} \geq0,\gamma_{it} \geq0,\delta_{it} \geq0 \; \forall i,t \\& \epsilon^-_{it} \geq0, \epsilon^+_{it} \geq 0 \; \forall i,t\end{split}\end{equation}
}\label{eq-1}\end{equation}

Equation~\ref{eq-1} is a probabilistic distance function, where the two
errors terms (\(\epsilon^-\) and \(\epsilon^+\)) allow for deviations
from the frontier, and \(\tau\) defines the quantile. We estimate the
model using a balanced panel of 105 countries for five years (2008-2012)
where the \(Z\) vector includes, trade to GDP ratio, the percentage of
the population which is urban, a dummy to the indicator if the country
is a target setting, and a set of year dummies. These environmental
variables adjust for observed cross-country and through time fluctuation
in the production technology. The estimated model results in performance
adjusted dual prices \(\gamma^{'}_{it},\beta^{'}_{it} ,\delta^{'}_{it}\)
which serve as inputs for the marginal abatement calculations. An
appealing feature of the specification in equation 1 is a separately
estimated intercept for each observation; \(\alpha_{it}\). These
intercept terms are analogous to random effects in hierarchical model
statistics, capturing unobserved time series and cross-sectional
variation.

{[}\^{}6{]}: We use GAMS software to encode our CQR and the CPLEX solver
to find an optimal solution.

\hypertarget{marginal-abatement}{%
\subsection{Marginal abatement}\label{marginal-abatement}}

Marginal abatement estimation uses a series of levels to find the local
quantile \(\tau^{*}\) for each observed data point. For example, a set
of ten quantile levels
\(\tau=(0.05,0.15,0.25,0.35,0.45,0.55,0.65,0.75,0.85,0.95)\)
{[}\^{}7{]}. In general, the number of quantiles is not fixed but should
depend on sample size and signal to noise ratio. Kuosmanen, Zhou, and
Dai (2020) note that in the traditional approach to shadow pricing using
frontier estimation, marginal abatement costs and shadow prices are
interchangeable terms. This is because prior approaches only use bad
output shadow prices measured in forgone good output units. They expand
the marginal abatement cost definition to include incremental use of
inputs by considering an optimal combination of shadow price
definitions:

\begin{enumerate}
\def\labelenumi{\arabic{enumi}.}
\tightlist
\item
  The marginal rate of transformation between good and bad outputs
  (MRT).
\item
  The marginal product of inputs on outputs (MP).
\end{enumerate}

In our study, we similarly calculate marginal abatement costs as:

\begin{enumerate}
\def\labelenumi{\arabic{enumi}.}
\tightlist
\item
  Find the largest expectile (\(\tau^{*}\)) for which the residual
  (\(\epsilon^+ + \epsilon^-\)) is non-negative.

  \begin{enumerate}
  \def\labelenumii{\roman{enumii}.}
  \tightlist
  \item
    For most observations, we find the nearest expectile by checking
    where the residual changes sign. For those observations, we take the
    weighted average of the shadow prices of the nearest executives,
    weighted by . For some observations, residuals are positive (or
    negative) for all executives (the best and the worst performers,
    respectively). For those, we use shadow prices of the highest/lowest
    expectile.
  \end{enumerate}
\item
  Calculate MRT and MP as the weighted average of quantiles for
  (\(\tau^{*}\)) and (\(\tau^{*+1}\)) weighted by the distance to the
  frontier of the quantiles(i.e.~the absolute value of the residuals).
  Specifically, these can be thought of as the sub derivatives with
  respect to the bad outputs from the distance function, where the
  marginal rate of substitution of output \$i\$ on bad output \(j\) is
  \(MRT_{\tau}(y_{i},b_{j})=-\frac{\delta \vec{D}_{\tau}/\delta b_{j}}{\delta \vec{D}_{\tau}/\delta y_{i}}\).
  Similarly the MP of input k on bad output j is
  \(MP_{\tau}(x_{k},b_{j})=\frac{\delta \vec{D}_{\tau}/\delta b_{j}}{\delta \vec{D}_{\tau}/\delta x_{k}}\)
\item
  Use the results from step 2, the marginal abatement cost (MAC) for bad
  output is defined as:
\item
  \begin{equation}\protect\hypertarget{eq-2}{}{
  \begin{equation}MAC(b_{j})=\displaystyle \min_{i,k}\{p_{i}MRT_{\tau}(y_{i},b_{j}), w_{k}MP_{\tau}(x_{k},b_{j})\}\end{equation}
  }\label{eq-2}\end{equation}

  \begin{enumerate}
  \def\labelenumii{\roman{enumii}.}
  \tightlist
  \item
    In Equation~\ref{eq-2} \(p_{i}\) is the price of output \(i\) and
    \(w_{k}\) is the price of input \(k\). This flexible definition of
    the MAC provides multiple opportunities for abatement. Specifically,
    bad output \(j\) can be abated by either reducing \emph{good}
    outputs (i.e., downscaling the GDP activity) or increasing the input
    use (for example, investment in the labour force or capital stock).
    This approach uses the least-cost alternative. In the case where the
    good outputs possess a monetary value, the sub derivatives (dual
    prices) provide monetary shadow prices for bad outputs, and the
    above equation simplifies to:
  \end{enumerate}
\end{enumerate}

\begin{equation}\protect\hypertarget{eq-3}{}{
\begin{equation}MAC(b_{j})=\displaystyle \min_{i,k}\{MRT_{\tau}(y_{i},b_{j}), MP_{\tau}(x_{k},b_{j})\} \end{equation}
}\label{eq-3}\end{equation}

In the above calculation, it is essential to ensure that the MRT and MP
enter the model simultaneously, given the scale of the inputs and
outputs entering the model. In our model, as both capital stock and GDP
enter the model in billions of dollars, the MRT and MP are directly
comparable in terms of minimum cost.

\hypertarget{application-of-statistical-test}{%
\subsection{Application of Statistical
Test}\label{application-of-statistical-test}}

In order to examine the impact of KP on environmental efficiency and
shadow prices (MACs) for CO\textsubscript{2} we want to look at group
differences between those countries who signed up to explicit emissions
reduction targets and those who did not. We utilize a test for group
differences in shadow prices first proposed by
(\textbf{Gallagher.2019?}).

Appendix A:1 details the theoretical exposition of shadow price group
difference testing .Suppose we have two series of the output ratio
y2/y1, representing two groups of firms observed in the same period or
the same sample of firms observed in two different periods. There are
several methods for testing whether the two series are significantly
different.

An obvious possibility is to apply a two-sample t-test for testing the
equality of means or the F-test for equal variances. This test requires
either that sample size is sufficiently large for asymptotic inferences
or that the ratio y2/y1 is normally distributed.

There are also several nonparametric alternatives. The (Wilcoxon)
Mann-Whitney U tests whether the medians of two independent
distributions are different. Another possibility is the two-sample
Kolmogorov-Smirnov test. If there is a pair of series(e.g., the same
firms observed in two different periods), then nonparametric rank-order
tests such as Spearman's rho and Kendall's tau can be used to test for
correlation between two series of y2/y1.

\hypertarget{testing-procedure}{%
\subsubsection{Testing procedure}\label{testing-procedure}}

There are three steps to the testing procedure for the difference in the
ratio series y2/y1. The first two steps are preliminary in that they
establish the statistical properties of the series, which informs the
choice of group difference test in the three-step.

\begin{enumerate}
\def\labelenumi{\arabic{enumi}.}
\item
  Test the empirical distribution of the series for normality. Whether
  the series is normally distributed determines whether a parametric or
  nonparametric test is needed. Stephens (1986) recommend the use of a
  normality test introduced by Anderson and Darling (1952) Anderson and
  Darling (1954). This procedure is a rank-sum test for goodness of fit
  based on the empirical distribution and has the advantage of giving
  more weight to the tails of the distribution.
\item
  Test the homogeneity of variance in the two groups. If step 1
  establishes normality, a simple F test of the homogeneity of variance
  can be performed. In the presence of non-normality, we turn to the
  Brown and Forsythe (1974) test, which extended the Levene (1961) ANOVA
  procedure applied to absolute deviations from the corresponding group
  mean. This Brown-Forsythe test transforms the variances into the
  absolute values of their deviations from the median. It uses a ratio
  of this transformed data as test statistics (See O'Brien (1981) for
  full explanation).
\item
  If the equal group variance and the normality assumptions are not
  rejected, then perform a Welch t-test for group mean differences
  (Welch 1947). The Kolmogorov-Smirnov nonparametric test provides a
  more robust statistical inference (Conover 1999). If only the
  normality assumption is rejected, the Wilcoxon Mann Whitney test is
  more appropriate.
\end{enumerate}

\hypertarget{references}{%
\section*{References}\label{references}}
\addcontentsline{toc}{section}{References}

\hypertarget{refs}{}
\begin{CSLReferences}{1}{0}
\leavevmode\vadjust pre{\hypertarget{ref-Anderson1952}{}}%
Anderson, T W, and D A Darling. 1952. {``Asymptotic Theory of Certain
{`Goodness of Fit'} Criteria Based on Stochastic Processes.''}
\emph{Ann. Math. Stat.} 23 (2): 193--212.

\leavevmode\vadjust pre{\hypertarget{ref-Anderson1954}{}}%
---------. 1954. {``A Test of Goodness of Fit.''} \emph{J. Am. Stat.
Assoc.} 49 (268): 765--69.

\leavevmode\vadjust pre{\hypertarget{ref-De_Angelis2019}{}}%
Angelis, Enrico Maria de, Marina Di Giacomo, and Davide Vannoni. 2019.
{``Climate Change and Economic Growth: The Role of Environmental Policy
Stringency.''} \emph{Sustain. Sci. Pract. Policy} 11 (8): 2273.

\leavevmode\vadjust pre{\hypertarget{ref-Brown1974}{}}%
Brown, Morton B, and Alan B Forsythe. 1974. {``Robust Tests for the
Equality of Variances.''} \emph{J. Am. Stat. Assoc.} 69 (346): 364--67.

\leavevmode\vadjust pre{\hypertarget{ref-Buonanno2003}{}}%
Buonanno, Paolo, Carlo Carraro, and Marzio Galeotti. 2003. {``Endogenous
Induced Technical Change and the Costs of Kyoto.''} \emph{Res. Energy
Econ.} 25 (1): 11--34.

\leavevmode\vadjust pre{\hypertarget{ref-Burniaux2000}{}}%
Burniaux, Jean-Marc. 2000. {``A Multi-Gas Assessment of the Kyoto
Protocol.''} \emph{OECD Economics Department Working Papers No. 270}.

\leavevmode\vadjust pre{\hypertarget{ref-Cifci2018}{}}%
Cifci, Eren, and Matthew E Oliver. 2018. {``Reassessing the Links
Between {GHG} Emissions, Economic Growth, and the {UNFCCC}: A
{Difference-in-Differences} Approach.''} \emph{Sustain. Sci. Pract.
Policy} 10 (2): 334.

\leavevmode\vadjust pre{\hypertarget{ref-Conover1999}{}}%
Conover, W J. 1999. \emph{Practical Nonparametric Statistics}. 3
edition. Wiley.

\leavevmode\vadjust pre{\hypertarget{ref-Dai.2020}{}}%
Dai, Sheng, Xun Zhou, and Timo Kuosmanen. 2020. {``{Forward-looking
assessment of the GHG abatement cost: Application to China}.''}
\emph{Energy Economics} 88: 104758.
\url{https://doi.org/10.1016/j.eneco.2020.104758}.

\leavevmode\vadjust pre{\hypertarget{ref-Fischer2006}{}}%
Fischer, Carolyn, and Richard D Morgenstern. 2006. {``Carbon Abatement
Costs: Why the Wide Range of Estimates?''} \emph{Energy J.} 27 (2):
73--86.

\leavevmode\vadjust pre{\hypertarget{ref-Halkos2014}{}}%
Halkos, George E, and Nickolaos G Tzeremes. 2014. {``Measuring the
Effect of Kyoto Protocol Agreement on Countries' Environmental
Efficiency in {Co2} Emissions: An Application of Conditional Full
Frontiers.''} \emph{J Prod Anal} 41 (3): 367--82.

\leavevmode\vadjust pre{\hypertarget{ref-Kumar.20207q}{}}%
Kumar, Surender, Shunsuke Managi, and Rakesh Kumar Jain. 2020. {``{CO2
mitigation policy for Indian thermal power sector: Potential gains from
emission trading}.''} \emph{Energy Economics} 86: 104653.
\url{https://doi.org/10.1016/j.eneco.2019.104653}.

\leavevmode\vadjust pre{\hypertarget{ref-Kuosmanen.2021}{}}%
Kuosmanen, Timo, and Xun Zhou. 2021a. {``{Shadow prices and marginal
abatement costs: Convex quantile regression approach}.''} \emph{European
Journal of Operational Research} 289 (2): 666--75.
\url{https://doi.org/10.1016/j.ejor.2020.07.036}.

\leavevmode\vadjust pre{\hypertarget{ref-Kousmanen.2021}{}}%
---------. 2021b. {``{Shadow prices and marginal abatement costs: Convex
quantile regression approach}.''} \emph{European Journal of Operational
Research} 289 (2): 666--75.
\url{https://doi.org/10.1016/j.ejor.2020.07.036}.

\leavevmode\vadjust pre{\hypertarget{ref-Kuosmanen.2020}{}}%
Kuosmanen, Timo, Xun Zhou, and Sheng Dai. 2020. {``{How much climate
policy has cost for OECD countries?}''} \emph{World Development} 125
(January): 104681. \url{https://doi.org/10.1016/j.worlddev.2019.104681}.

\leavevmode\vadjust pre{\hypertarget{ref-Levene1961}{}}%
Levene, Howard. 1961. {``Robust Tests for Equality of Variances.''}
\emph{Contributions to Probability and Statistics. Essays in Honor of
Harold Hotelling}, 279--92.

\leavevmode\vadjust pre{\hypertarget{ref-McKibbin2004}{}}%
McKibbin, Warwick J, and Peter J Wilcoxen. 2004. {``Estimates of the
Costs of Kyoto: Marrakesh Versus the {McKibbin--Wilcoxen} Blueprint.''}
\emph{Energy Policy} 32 (4): 467--79.

\leavevmode\vadjust pre{\hypertarget{ref-Nordhaus1999}{}}%
Nordhaus, William D, and Joseph G Boyer. 1999. {``Requiem for Kyoto: An
Economic Analysis of the Kyoto Protocol.''} \emph{Energy J.} 20:
93--130.

\leavevmode\vadjust pre{\hypertarget{ref-OBrien1981}{}}%
O'Brien, Ralph G. 1981. {``A Simple Test for Variance Effects in
Experimental Designs.''} \emph{Psychol. Bull.} 89 (3): 570--74.

\leavevmode\vadjust pre{\hypertarget{ref-Reilly1999}{}}%
Reilly, J, R Prinn, J Harnisch, J Fitzmaurice, H Jacoby, D Kicklighter,
J Melillo, P Stone, A Sokolov, and C Wang. 1999. {``Multi-Gas Assessment
of the Kyoto Protocol.''} \emph{Nature} 401 (6753): 549--55.

\leavevmode\vadjust pre{\hypertarget{ref-Stephens1986}{}}%
Stephens, Michael A. 1986. {``Tests Based on {EDF} Statistics.''} In
\emph{Goodness-of-Fit-Techniques}, 97--194. Routledge.

\leavevmode\vadjust pre{\hypertarget{ref-Tsionas.2022}{}}%
Tsionas, Mike G. 2022. {``{Convex Non-Parametric Least Squares, Causal
Structures and Productivity}.''} \emph{European Journal of Operational
Research}. \url{https://doi.org/10.1016/j.ejor.2022.02.020}.

\leavevmode\vadjust pre{\hypertarget{ref-Vale2016}{}}%
Vale, Petterson Molina. 2016. {``The Changing Climate of Climate Change
Economics.''} \emph{Ecol. Econ.} 121 (January): 12--19.

\leavevmode\vadjust pre{\hypertarget{ref-Welch1947}{}}%
Welch, B L. 1947. {``The Generalisation of Student's Problems When
Several Different Population Variances Are Involved.''}
\emph{Biometrika} 34 (1-2): 28--35.

\leavevmode\vadjust pre{\hypertarget{ref-Xian.2022}{}}%
Xian, Yujiao, Dan Yu, Ke Wang, Jian Yu, and Zhimin Huang. 2022.
{``{Capturing the least costly measure of CO2 emission abatement:
Evidence from the iron and steel industry in China}.''} \emph{Energy
Economics} 106: 105812.
\url{https://doi.org/10.1016/j.eneco.2022.105812}.

\leavevmode\vadjust pre{\hypertarget{ref-Zhang1998}{}}%
Zhang, Zhongxiang, and Henk Folmer. 1998. {``Economic Modelling
Approaches to Cost Estimates for the Control of Carbon Dioxide
Emissions.''} \emph{Energy Econ.} 20 (1): 101--20.

\end{CSLReferences}



\end{document}
