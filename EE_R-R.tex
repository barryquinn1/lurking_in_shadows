% Options for packages loaded elsewhere
\PassOptionsToPackage{unicode}{hyperref}
\PassOptionsToPackage{hyphens}{url}
\PassOptionsToPackage{dvipsnames,svgnames,x11names}{xcolor}
%
\documentclass[
]{article}\usepackage{amsmath,amssymb}
\usepackage{lmodern}
\usepackage{iftex}
\ifPDFTeX
  \usepackage[T1]{fontenc}
  \usepackage[utf8]{inputenc}
  \usepackage{textcomp} % provide euro and other symbols
\else % if luatex or xetex
  \usepackage{unicode-math}
  \defaultfontfeatures{Scale=MatchLowercase}
  \defaultfontfeatures[\rmfamily]{Ligatures=TeX,Scale=1}
\fi
% Use upquote if available, for straight quotes in verbatim environments
\IfFileExists{upquote.sty}{\usepackage{upquote}}{}
\IfFileExists{microtype.sty}{% use microtype if available
  \usepackage[]{microtype}
  \UseMicrotypeSet[protrusion]{basicmath} % disable protrusion for tt fonts
}{}
\makeatletter
\@ifundefined{KOMAClassName}{% if non-KOMA class
  \IfFileExists{parskip.sty}{%
    \usepackage{parskip}
  }{% else
    \setlength{\parindent}{0pt}
    \setlength{\parskip}{6pt plus 2pt minus 1pt}}
}{% if KOMA class
  \KOMAoptions{parskip=half}}
\makeatother
\usepackage{xcolor}
\IfFileExists{xurl.sty}{\usepackage{xurl}}{} % add URL line breaks if available
\IfFileExists{bookmark.sty}{\usepackage{bookmark}}{\usepackage{hyperref}}
\hypersetup{
  pdftitle={Lurking in the shadows: The impact of CO2 emissions target setting on carbon pricing and environmental efficiency.},
  pdfauthor={Barry Quinn (Queen's Management School); Ronan Gallagher (University of Edinburgh Business School); Timo Kuosmanen (Aalto Business School)},
  colorlinks=true,
  linkcolor={blue},
  filecolor={Maroon},
  citecolor={Blue},
  urlcolor={Blue},
  pdfcreator={LaTeX via pandoc}}
\urlstyle{same} % disable monospaced font for URLs
\setlength{\emergencystretch}{3em} % prevent overfull lines
\setcounter{secnumdepth}{-\maxdimen} % remove section numbering
% Make \paragraph and \subparagraph free-standing
\ifx\paragraph\undefined\else
  \let\oldparagraph\paragraph
  \renewcommand{\paragraph}[1]{\oldparagraph{#1}\mbox{}}
\fi
\ifx\subparagraph\undefined\else
  \let\oldsubparagraph\subparagraph
  \renewcommand{\subparagraph}[1]{\oldsubparagraph{#1}\mbox{}}
\fi


\providecommand{\tightlist}{%
  \setlength{\itemsep}{0pt}\setlength{\parskip}{0pt}}\usepackage{longtable,booktabs,array}
\usepackage{calc} % for calculating minipage widths
% Correct order of tables after \paragraph or \subparagraph
\usepackage{etoolbox}
\makeatletter
\patchcmd\longtable{\par}{\if@noskipsec\mbox{}\fi\par}{}{}
\makeatother
% Allow footnotes in longtable head/foot
\IfFileExists{footnotehyper.sty}{\usepackage{footnotehyper}}{\usepackage{footnote}}
\makesavenoteenv{longtable}
\usepackage{graphicx}
\makeatletter
\def\maxwidth{\ifdim\Gin@nat@width>\linewidth\linewidth\else\Gin@nat@width\fi}
\def\maxheight{\ifdim\Gin@nat@height>\textheight\textheight\else\Gin@nat@height\fi}
\makeatother
% Scale images if necessary, so that they will not overflow the page
% margins by default, and it is still possible to overwrite the defaults
% using explicit options in \includegraphics[width, height, ...]{}
\setkeys{Gin}{width=\maxwidth,height=\maxheight,keepaspectratio}
% Set default figure placement to htbp
\makeatletter
\def\fps@figure{htbp}
\makeatother
\newlength{\cslhangindent}
\setlength{\cslhangindent}{1.5em}
\newlength{\csllabelwidth}
\setlength{\csllabelwidth}{3em}
\newlength{\cslentryspacingunit} % times entry-spacing
\setlength{\cslentryspacingunit}{\parskip}
\newenvironment{CSLReferences}[2] % #1 hanging-ident, #2 entry spacing
 {% don't indent paragraphs
  \setlength{\parindent}{0pt}
  % turn on hanging indent if param 1 is 1
  \ifodd #1
  \let\oldpar\par
  \def\par{\hangindent=\cslhangindent\oldpar}
  \fi
  % set entry spacing
  \setlength{\parskip}{#2\cslentryspacingunit}
 }%
 {}
\usepackage{calc}
\newcommand{\CSLBlock}[1]{#1\hfill\break}
\newcommand{\CSLLeftMargin}[1]{\parbox[t]{\csllabelwidth}{#1}}
\newcommand{\CSLRightInline}[1]{\parbox[t]{\linewidth - \csllabelwidth}{#1}\break}
\newcommand{\CSLIndent}[1]{\hspace{\cslhangindent}#1}

\makeatletter
\makeatother
\makeatletter
\@ifpackageloaded{caption}{}{\usepackage{caption}}
\AtBeginDocument{%
\ifdefined\contentsname
  \renewcommand*\contentsname{Table of contents}
\else
  \newcommand\contentsname{Table of contents}
\fi
\ifdefined\listfigurename
  \renewcommand*\listfigurename{List of Figures}
\else
  \newcommand\listfigurename{List of Figures}
\fi
\ifdefined\listtablename
  \renewcommand*\listtablename{List of Tables}
\else
  \newcommand\listtablename{List of Tables}
\fi
\ifdefined\figurename
  \renewcommand*\figurename{Figure}
\else
  \newcommand\figurename{Figure}
\fi
\ifdefined\tablename
  \renewcommand*\tablename{Table}
\else
  \newcommand\tablename{Table}
\fi
}
\@ifpackageloaded{float}{}{\usepackage{float}}
\floatstyle{ruled}
\@ifundefined{c@chapter}{\newfloat{codelisting}{h}{lop}}{\newfloat{codelisting}{h}{lop}[chapter]}
\floatname{codelisting}{Listing}
\newcommand*\listoflistings{\listof{codelisting}{List of Listings}}
\makeatother
\makeatletter
\@ifpackageloaded{caption}{}{\usepackage{caption}}
\@ifpackageloaded{subcaption}{}{\usepackage{subcaption}}
\makeatother
\makeatletter
\@ifpackageloaded{tcolorbox}{}{\usepackage[many]{tcolorbox}}
\makeatother
\makeatletter
\@ifundefined{shadecolor}{\definecolor{shadecolor}{rgb}{.97, .97, .97}}
\makeatother
\makeatletter
\makeatother
\ifLuaTeX
  \usepackage{selnolig}  % disable illegal ligatures
\fi

\title{\emph{Lurking in the shadows:} The impact of CO\textsubscript{2}
emissions target setting on carbon pricing and environmental
efficiency.}
\author{Barry Quinn (Queen's Management School) \and Ronan Gallagher
(University of Edinburgh Business School) \and Timo Kuosmanen (Aalto
Business School)}
\date{}

\begin{document}
\maketitle

\ifdefined\Shaded\renewenvironment{Shaded}{\begin{tcolorbox}[enhanced, interior hidden, breakable, boxrule=0pt, sharp corners, borderline west={3pt}{0pt}{shadecolor}, frame hidden]}{\end{tcolorbox}}\fi

\hypertarget{abstract}{%
\subsection{Abstract}\label{abstract}}

This paper studies the impact of CO\textsubscript{2} emissions target
setting. We empirically investigate the targets set during the Kyoto
Protocol period using a convex non-parametric least squares system,
quantile regressions, and a comprehensive data set of 125 countries. Our
findings reveal CO\textsubscript{2} marginal abatement costs, which: (1)
are significantly higher for target setting countries; (2) increase over
the sample period; (3) and are an order of magnitude greater than the
prevailing emissions pricing mechanisms. The results provide insights
into the consequences of policies to curb unwanted by-products in a
regulated system and shed light on the price efficiency of carbon
markets. Furthermore, we contribute to the debate on emission reduction
standard-setting and highlight the importance of shadow price estimates
when regulating market instabilities in an emission trading scheme.

\newpage

\hypertarget{introduction}{%
\subsection{Introduction}\label{introduction}}

The World Health Organization predicts significant health risks
associated with climate change. Their analysis estimates around 250,000
additional deaths per year from 2030 to 2050, assuming the status quo of
current abatement practices and global economic growth\footnote{These
  statistics are taken from the WHO factsheet on climate change and
  health
  \url{https://www.who.int/news-room/fact-sheets/detail/climate-change-and-health}}.
The reduction of greenhouse emissions and the impact on climate change
are existential challenges of the 21st century. Many countries have
adopted emissions reductions targets since the dawn of the Kyoto
Protocol (hereafter KP)\footnote{The KP was principled on the idea of
  hard targets for emissions reduction for industrialised nations and
  the EU. These were developed in tandem with carbon trading mechanisms,
  the largest of which is the EU Emissions Trading System (EU-ETS).
  These developments brought not just matters of environmental
  production efficiency to the fore but also those relating to carbon
  price discovery.} in response to this challenge. The nature and
efficacy in response to this challenge of these targets have attracted
considerable conceptual debate (for example, see Angelis, Di Giacomo,
and Vannoni (2019) ), but few empirical studies on the impact of
explicit target setting.

We attempt to solve this puzzle by testing the differences between
target-setting and non-target-setting countries. We use an
identification strategy that more accurately estimates the impact of
target setting on carbon pricing and environmental efficiency.
Specifically, we focus on the KP target setting period and analyse
CO\textsubscript{2} emissions for 125 countries. Considering only
CO\textsubscript{2} emissions allow for a representative sample of
non-target setting Countries (non-annexed 1 Countries) and more
meaningful group comparisons in our statistical tests. We use quantile
system of convex nonparametric least squares regressions (CQR) to
estimate shadow prices (see equation (1) in Kuosmanen and Zhou (2021a) )
and an improved marginal abatement cost (MAC) of CO\textsubscript{2}
emissions (Xian et al. 2022; Dai, Zhou, and Kuosmanen 2020; Kuosmanen,
Zhou, and Dai 2020; Kuosmanen and Zhou 2021b). Convex nonparametric
least squares has recently been found to admit a causal interpretation
between inefficiency and productivity (Tsionas 2022). Moreover, our
method allows for an examination of the factors which help explain
relative (in)efficiencies.

We find that target setters during the first KP commitment period were
more environmentally inefficient than non-target setters, an unintended
consequence of the regulation. We also note that countries with a higher
degree of industrialisation and those with more urban populations
exhibit lower environmental efficiency. Our results also assert that the
marginal cost of CO2 reduction during the first KP period was an order
of magnitude higher than the trading price of CO2 in the EU-ETS. This
result suggests considerable price inefficiency in the emissions market.

Our findings have important implications for international carbon
regulation. Authors have highlighted potential gains in CO2 mitigation
from emission trading schemes (for example, Kumar, Managi, and Jain
2020). Our findings add to this debate. We show that the shadow prices
and market prices of CO2 diverge in the KP period, suggestive of a
consistent misallocation of the traded allowances in the EU emissions
trading scheme (ETS). Our results also show an imbalance in shadow
pricing due to target setting. These significant frictions in the price
discovery of an ETS market may result in a surplus of allowances,
exacerbating market instability, and a lower carbon price. The latter
likely weakened the incentives to lower emissions. We argue that when
policymakers debate structural measures to promote market stability,
such as predefined rules to place unallocated allowances in a market
stability reserve, shadow price imbalances due to target setting must be
considered{[}3{]}.

In the next section, we review the literature on the impact of the KP on
emissions and productive efficiency. Next, we describe the frontier
models and data used. We follow with a discussion of our findings and
conclusions.

\hypertarget{literature-review}{%
\section{Literature Review}\label{literature-review}}

The academic inquiry into the effective management of climate change has
a rich history. Historically, holistic models seek to understand how
human development, societal choices, and the natural world integrate and
influence each other. At a simplistic level, they can estimate the
social cost of carbon pollutants. This top-down approach to the
economics of climate change has been at the forefront of the discipline

. However, such a global approach may prove dated in the face of stalled
international coordination on climate change policy.

Against the bedrock of climate science, the KP agreement was an
ambitious attempt to coordinate across borders on targets for emissions
reduction. The KP set out to differentiate reduction targets equitably
in terms of a nation's industrial development, a comparable level of
pollution, and the ability to mitigate the ecological damage of global
emissions levels. Specifically, countries were categorised into two
Annexes. Annexe 2 countries, which set explicit targets, were mostly
developed nations, with higher industrial production. Annexe 1
countries, defined as developing, were not subject to targets, although
most ratified the Protocol.

\hypertarget{refs}{}
\begin{CSLReferences}{1}{0}
\leavevmode\vadjust pre{\hypertarget{ref-De_Angelis2019}{}}%
Angelis, Enrico Maria de, Marina Di Giacomo, and Davide Vannoni. 2019.
{``Climate Change and Economic Growth: The Role of Environmental Policy
Stringency.''} \emph{Sustain. Sci. Pract. Policy} 11 (8): 2273.

\leavevmode\vadjust pre{\hypertarget{ref-Dai.2020}{}}%
Dai, Sheng, Xun Zhou, and Timo Kuosmanen. 2020. {``{Forward-looking
assessment of the GHG abatement cost: Application to China}.''}
\emph{Energy Economics} 88: 104758.
\url{https://doi.org/10.1016/j.eneco.2020.104758}.

\leavevmode\vadjust pre{\hypertarget{ref-Kuosmanen.2021}{}}%
Kuosmanen, Timo, and Xun Zhou. 2021a. {``{Shadow prices and marginal
abatement costs: Convex quantile regression approach}.''} \emph{European
Journal of Operational Research} 289 (2): 666--75.
\url{https://doi.org/10.1016/j.ejor.2020.07.036}.

\leavevmode\vadjust pre{\hypertarget{ref-Kousmanen.2021}{}}%
---------. 2021b. {``{Shadow prices and marginal abatement costs: Convex
quantile regression approach}.''} \emph{European Journal of Operational
Research} 289 (2): 666--75.
\url{https://doi.org/10.1016/j.ejor.2020.07.036}.

\leavevmode\vadjust pre{\hypertarget{ref-Kuosmanen.2020}{}}%
Kuosmanen, Timo, Xun Zhou, and Sheng Dai. 2020. {``{How much climate
policy has cost for OECD countries?}''} \emph{World Development} 125
(January): 104681. \url{https://doi.org/10.1016/j.worlddev.2019.104681}.

\leavevmode\vadjust pre{\hypertarget{ref-Tsionas.2022}{}}%
Tsionas, Mike G. 2022. {``{Convex Non-Parametric Least Squares, Causal
Structures and Productivity}.''} \emph{European Journal of Operational
Research}. \url{https://doi.org/10.1016/j.ejor.2022.02.020}.

\leavevmode\vadjust pre{\hypertarget{ref-Xian.2022}{}}%
Xian, Yujiao, Dan Yu, Ke Wang, Jian Yu, and Zhimin Huang. 2022.
{``{Capturing the least costly measure of CO2 emission abatement:
Evidence from the iron and steel industry in China}.''} \emph{Energy
Economics} 106: 105812.
\url{https://doi.org/10.1016/j.eneco.2022.105812}.

\end{CSLReferences}


\end{document}
